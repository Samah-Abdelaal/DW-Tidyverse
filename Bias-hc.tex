\documentclass[]{article}
\usepackage{lmodern}
\usepackage{amssymb,amsmath}
\usepackage{ifxetex,ifluatex}
\usepackage{fixltx2e} % provides \textsubscript
\ifnum 0\ifxetex 1\fi\ifluatex 1\fi=0 % if pdftex
  \usepackage[T1]{fontenc}
  \usepackage[utf8]{inputenc}
\else % if luatex or xelatex
  \ifxetex
    \usepackage{mathspec}
  \else
    \usepackage{fontspec}
  \fi
  \defaultfontfeatures{Ligatures=TeX,Scale=MatchLowercase}
\fi
% use upquote if available, for straight quotes in verbatim environments
\IfFileExists{upquote.sty}{\usepackage{upquote}}{}
% use microtype if available
\IfFileExists{microtype.sty}{%
\usepackage{microtype}
\UseMicrotypeSet[protrusion]{basicmath} % disable protrusion for tt fonts
}{}
\usepackage[margin=1in]{geometry}
\usepackage{hyperref}
\hypersetup{unicode=true,
            pdftitle={Bias in Healthcare},
            pdfborder={0 0 0},
            breaklinks=true}
\urlstyle{same}  % don't use monospace font for urls
\usepackage{color}
\usepackage{fancyvrb}
\newcommand{\VerbBar}{|}
\newcommand{\VERB}{\Verb[commandchars=\\\{\}]}
\DefineVerbatimEnvironment{Highlighting}{Verbatim}{commandchars=\\\{\}}
% Add ',fontsize=\small' for more characters per line
\usepackage{framed}
\definecolor{shadecolor}{RGB}{248,248,248}
\newenvironment{Shaded}{\begin{snugshade}}{\end{snugshade}}
\newcommand{\AlertTok}[1]{\textcolor[rgb]{0.94,0.16,0.16}{#1}}
\newcommand{\AnnotationTok}[1]{\textcolor[rgb]{0.56,0.35,0.01}{\textbf{\textit{#1}}}}
\newcommand{\AttributeTok}[1]{\textcolor[rgb]{0.77,0.63,0.00}{#1}}
\newcommand{\BaseNTok}[1]{\textcolor[rgb]{0.00,0.00,0.81}{#1}}
\newcommand{\BuiltInTok}[1]{#1}
\newcommand{\CharTok}[1]{\textcolor[rgb]{0.31,0.60,0.02}{#1}}
\newcommand{\CommentTok}[1]{\textcolor[rgb]{0.56,0.35,0.01}{\textit{#1}}}
\newcommand{\CommentVarTok}[1]{\textcolor[rgb]{0.56,0.35,0.01}{\textbf{\textit{#1}}}}
\newcommand{\ConstantTok}[1]{\textcolor[rgb]{0.00,0.00,0.00}{#1}}
\newcommand{\ControlFlowTok}[1]{\textcolor[rgb]{0.13,0.29,0.53}{\textbf{#1}}}
\newcommand{\DataTypeTok}[1]{\textcolor[rgb]{0.13,0.29,0.53}{#1}}
\newcommand{\DecValTok}[1]{\textcolor[rgb]{0.00,0.00,0.81}{#1}}
\newcommand{\DocumentationTok}[1]{\textcolor[rgb]{0.56,0.35,0.01}{\textbf{\textit{#1}}}}
\newcommand{\ErrorTok}[1]{\textcolor[rgb]{0.64,0.00,0.00}{\textbf{#1}}}
\newcommand{\ExtensionTok}[1]{#1}
\newcommand{\FloatTok}[1]{\textcolor[rgb]{0.00,0.00,0.81}{#1}}
\newcommand{\FunctionTok}[1]{\textcolor[rgb]{0.00,0.00,0.00}{#1}}
\newcommand{\ImportTok}[1]{#1}
\newcommand{\InformationTok}[1]{\textcolor[rgb]{0.56,0.35,0.01}{\textbf{\textit{#1}}}}
\newcommand{\KeywordTok}[1]{\textcolor[rgb]{0.13,0.29,0.53}{\textbf{#1}}}
\newcommand{\NormalTok}[1]{#1}
\newcommand{\OperatorTok}[1]{\textcolor[rgb]{0.81,0.36,0.00}{\textbf{#1}}}
\newcommand{\OtherTok}[1]{\textcolor[rgb]{0.56,0.35,0.01}{#1}}
\newcommand{\PreprocessorTok}[1]{\textcolor[rgb]{0.56,0.35,0.01}{\textit{#1}}}
\newcommand{\RegionMarkerTok}[1]{#1}
\newcommand{\SpecialCharTok}[1]{\textcolor[rgb]{0.00,0.00,0.00}{#1}}
\newcommand{\SpecialStringTok}[1]{\textcolor[rgb]{0.31,0.60,0.02}{#1}}
\newcommand{\StringTok}[1]{\textcolor[rgb]{0.31,0.60,0.02}{#1}}
\newcommand{\VariableTok}[1]{\textcolor[rgb]{0.00,0.00,0.00}{#1}}
\newcommand{\VerbatimStringTok}[1]{\textcolor[rgb]{0.31,0.60,0.02}{#1}}
\newcommand{\WarningTok}[1]{\textcolor[rgb]{0.56,0.35,0.01}{\textbf{\textit{#1}}}}
\usepackage{graphicx}
% grffile has become a legacy package: https://ctan.org/pkg/grffile
\IfFileExists{grffile.sty}{%
\usepackage{grffile}
}{}
\makeatletter
\def\maxwidth{\ifdim\Gin@nat@width>\linewidth\linewidth\else\Gin@nat@width\fi}
\def\maxheight{\ifdim\Gin@nat@height>\textheight\textheight\else\Gin@nat@height\fi}
\makeatother
% Scale images if necessary, so that they will not overflow the page
% margins by default, and it is still possible to overwrite the defaults
% using explicit options in \includegraphics[width, height, ...]{}
\setkeys{Gin}{width=\maxwidth,height=\maxheight,keepaspectratio}
\IfFileExists{parskip.sty}{%
\usepackage{parskip}
}{% else
\setlength{\parindent}{0pt}
\setlength{\parskip}{6pt plus 2pt minus 1pt}
}
\setlength{\emergencystretch}{3em}  % prevent overfull lines
\providecommand{\tightlist}{%
  \setlength{\itemsep}{0pt}\setlength{\parskip}{0pt}}
\setcounter{secnumdepth}{0}
% Redefines (sub)paragraphs to behave more like sections
\ifx\paragraph\undefined\else
\let\oldparagraph\paragraph
\renewcommand{\paragraph}[1]{\oldparagraph{#1}\mbox{}}
\fi
\ifx\subparagraph\undefined\else
\let\oldsubparagraph\subparagraph
\renewcommand{\subparagraph}[1]{\oldsubparagraph{#1}\mbox{}}
\fi

%%% Use protect on footnotes to avoid problems with footnotes in titles
\let\rmarkdownfootnote\footnote%
\def\footnote{\protect\rmarkdownfootnote}

%%% Change title format to be more compact
\usepackage{titling}

% Create subtitle command for use in maketitle
\providecommand{\subtitle}[1]{
  \posttitle{
    \begin{center}\large#1\end{center}
    }
}

\setlength{\droptitle}{-2em}

  \title{Bias in Healthcare}
    \pretitle{\vspace{\droptitle}\centering\huge}
  \posttitle{\par}
    \author{}
    \preauthor{}\postauthor{}
    \date{}
    \predate{}\postdate{}
  

\begin{document}
\maketitle

\hypertarget{setup}{%
\section{Setup}\label{setup}}

\hypertarget{load-packages}{%
\subsubsection{Load packages}\label{load-packages}}

\begin{Shaded}
\begin{Highlighting}[]
\KeywordTok{library}\NormalTok{(tidyverse)}
\end{Highlighting}
\end{Shaded}

\begin{verbatim}
## -- Attaching packages ------------------------------- tidyverse 1.2.1 --
\end{verbatim}

\begin{verbatim}
## v ggplot2 3.2.1     v purrr   0.3.3
## v tibble  2.1.3     v dplyr   0.8.3
## v tidyr   1.0.0     v stringr 1.4.0
## v readr   1.3.1     v forcats 0.4.0
\end{verbatim}

\begin{verbatim}
## -- Conflicts ---------------------------------- tidyverse_conflicts() --
## x dplyr::filter() masks stats::filter()
## x dplyr::lag()    masks stats::lag()
\end{verbatim}

\begin{Shaded}
\begin{Highlighting}[]
\KeywordTok{library}\NormalTok{(ggplot2)}
\KeywordTok{library}\NormalTok{(ggthemes)}
\KeywordTok{library}\NormalTok{(scales)}
\end{Highlighting}
\end{Shaded}

\begin{verbatim}
## 
## Attaching package: 'scales'
\end{verbatim}

\begin{verbatim}
## The following object is masked from 'package:purrr':
## 
##     discard
\end{verbatim}

\begin{verbatim}
## The following object is masked from 'package:readr':
## 
##     col_factor
\end{verbatim}

\begin{center}\rule{0.5\linewidth}{\linethickness}\end{center}

\begin{center}\rule{0.5\linewidth}{\linethickness}\end{center}

\hypertarget{part-1-data}{%
\subsection{Part 1: Data}\label{part-1-data}}

The Behavioral Risk Factor Surveillance System (BRFSS) is a telephone
survey that collects data about U.S. residents regarding their
health-related risk behaviors, chronic health conditions, and use of
preventive services. It collects data in all 50 states as well as the
District of Columbia and three U.S. territories. BRFSS completes more
than 400,000 adult interviews each year, making it the largest
continuously conducted health survey system in the world.

Population: Health characteristics estimated from the BRFSS pertain to
the non-institutionalized adult population, aged 18 years or older, who
reside in the US.

Respondent data are forwarded to CDC to be aggregated for each state,
returned with standard tabulations, and published at year's end by each
state. Source: \url{https://www.cdc.gov/brfss/}

In this project, I chose only the variables concerning race and its
effect on the experience of seeking healthcare.The purpose of this
analysis is to detect bias in healthcare delivery in the USA.

\hypertarget{load-data}{%
\subsubsection{Load data}\label{load-data}}

\begin{Shaded}
\begin{Highlighting}[]
\NormalTok{race_data1 <-}\StringTok{ }\KeywordTok{read_csv}\NormalTok{(}\StringTok{"Data/race_data1.csv"}\NormalTok{)}
\end{Highlighting}
\end{Shaded}

\begin{verbatim}
## Parsed with column specification:
## cols(
##   rrclass2 = col_character(),
##   rrhcare3 = col_character()
## )
\end{verbatim}

\hypertarget{viewing-the-structure-of-the-data}{%
\section{Viewing the structure of the
data}\label{viewing-the-structure-of-the-data}}

\begin{Shaded}
\begin{Highlighting}[]
\KeywordTok{str}\NormalTok{(race_data1)}
\end{Highlighting}
\end{Shaded}

\begin{verbatim}
## Classes 'spec_tbl_df', 'tbl_df', 'tbl' and 'data.frame': 2658 obs. of  2 variables:
##  $ rrclass2: chr  "Black or African American" "White" "White" "White" ...
##  $ rrhcare3: chr  "The same as other races" "The same as other races" "The same as other races" "The same as other races" ...
##  - attr(*, "spec")=
##   .. cols(
##   ..   rrclass2 = col_character(),
##   ..   rrhcare3 = col_character()
##   .. )
\end{verbatim}

\textbf{rrclass2}: How Do Other People Usually Classify You In This
Country?

\begin{Shaded}
\begin{Highlighting}[]
\KeywordTok{class}\NormalTok{(race_data1}\OperatorTok{$}\NormalTok{rrclass2)}
\end{Highlighting}
\end{Shaded}

\begin{verbatim}
## [1] "character"
\end{verbatim}

\begin{Shaded}
\begin{Highlighting}[]
\KeywordTok{unique}\NormalTok{(race_data1}\OperatorTok{$}\NormalTok{rrclass2)}
\end{Highlighting}
\end{Shaded}

\begin{verbatim}
## [1] "Black or African American" "White"
\end{verbatim}

\textbf{rrhcare3}: When Seeking Health Care Past 12 Months, Was
Experience Worse, Same, Better Than other races

\begin{Shaded}
\begin{Highlighting}[]
\KeywordTok{class}\NormalTok{(race_data1}\OperatorTok{$}\NormalTok{rrhcare3)}
\end{Highlighting}
\end{Shaded}

\begin{verbatim}
## [1] "character"
\end{verbatim}

\begin{Shaded}
\begin{Highlighting}[]
\KeywordTok{unique}\NormalTok{(race_data1}\OperatorTok{$}\NormalTok{rrhcare3)}
\end{Highlighting}
\end{Shaded}

\begin{verbatim}
## [1] "The same as other races" "Better than other races"
\end{verbatim}

\begin{center}\rule{0.5\linewidth}{\linethickness}\end{center}

\begin{center}\rule{0.5\linewidth}{\linethickness}\end{center}

\hypertarget{data-summarization}{%
\section{Data Summarization}\label{data-summarization}}

\begin{Shaded}
\begin{Highlighting}[]
\NormalTok{race_data1 }\OperatorTok
\StringTok{  }\KeywordTok{group_by}\NormalTok{(rrclass2) }\OperatorTok
\StringTok{  }\KeywordTok{summarize}\NormalTok{(}\DataTypeTok{count =} \KeywordTok{n}\NormalTok{()) ->}\StringTok{ }\NormalTok{count_tab}
\NormalTok{pct <-}\StringTok{ }\KeywordTok{list}\NormalTok{(}\KeywordTok{c}\NormalTok{(}\KeywordTok{paste0}\NormalTok{(}\KeywordTok{round}\NormalTok{((}\DecValTok{168}\OperatorTok{/}\KeywordTok{length}\NormalTok{(race_data1}\OperatorTok{$}\NormalTok{rrclass2))}\OperatorTok{*}\DecValTok{100}\NormalTok{), }\StringTok{"%"}\NormalTok{),                           }\KeywordTok{paste0}\NormalTok{(}\KeywordTok{round}\NormalTok{((}\DecValTok{2490}\OperatorTok{/}\KeywordTok{length}\NormalTok{(race_data1}\OperatorTok{$}\NormalTok{rrclass2))}\OperatorTok{*}\DecValTok{100}\NormalTok{),}\StringTok{"%"}\NormalTok{)))}
\NormalTok{freq_tab <-}\StringTok{ }\KeywordTok{cbind}\NormalTok{(count_tab, pct)}
\KeywordTok{colnames}\NormalTok{(freq_tab) <-}\StringTok{ }\KeywordTok{c}\NormalTok{(}\StringTok{"Race"}\NormalTok{, }\StringTok{"Frequency"}\NormalTok{, }\StringTok{" Percentage"}\NormalTok{)}
\NormalTok{freq_tab}
\end{Highlighting}
\end{Shaded}

\begin{verbatim}
##                        Race Frequency  Percentage
## 1 Black or African American       168          6%
## 2                     White      2490         94%
\end{verbatim}

As seen in the above frequency table, the majority of our sample is
White \textbf{(94\%)}.

\hypertarget{cross-tabulating-our-data}{%
\subsubsection{Cross-tabulating our
data}\label{cross-tabulating-our-data}}

\begin{Shaded}
\begin{Highlighting}[]
\NormalTok{tab1 <-}\StringTok{ }\KeywordTok{xtabs}\NormalTok{(}\OperatorTok{~}\StringTok{ }\NormalTok{rrclass2 }\OperatorTok{+}\StringTok{ }\NormalTok{rrhcare3, }\DataTypeTok{data =}\NormalTok{ race_data1)}
\NormalTok{tab1}
\end{Highlighting}
\end{Shaded}

\begin{verbatim}
##                            rrhcare3
## rrclass2                    Better than other races The same as other races
##   Black or African American                      17                     151
##   White                                         370                    2120
\end{verbatim}

\begin{center}\rule{0.5\linewidth}{\linethickness}\end{center}

\begin{center}\rule{0.5\linewidth}{\linethickness}\end{center}

\hypertarget{using-visualization-to-find}{%
\section{Using visualization to
find:}\label{using-visualization-to-find}}

Which race has a higher percentage of people having a better healthcare
experience than other races compared with those having the same
experience as other races.

\hypertarget{for-the-white-race}{%
\subsubsection{For the white race}\label{for-the-white-race}}

\begin{Shaded}
\begin{Highlighting}[]
\NormalTok{race_white <-}\StringTok{ }\NormalTok{race_data1 }\OperatorTok
\StringTok{  }\KeywordTok{filter}\NormalTok{(rrclass2}\OperatorTok{==}\StringTok{"White"}\NormalTok{)}\OperatorTok
\StringTok{  }\KeywordTok{group_by}\NormalTok{(rrhcare3) }\OperatorTok
\StringTok{  }\KeywordTok{summarise}\NormalTok{(}\DataTypeTok{counts =} \KeywordTok{n}\NormalTok{())}
\end{Highlighting}
\end{Shaded}

\begin{Shaded}
\begin{Highlighting}[]
\NormalTok{race_white }\OperatorTok
\StringTok{  }\KeywordTok{ggplot}\NormalTok{(}\KeywordTok{aes}\NormalTok{(}\DataTypeTok{x=}\NormalTok{ rrhcare3,}
             \DataTypeTok{y=}\NormalTok{ counts}\OperatorTok{/}\KeywordTok{sum}\NormalTok{(counts),}
             \DataTypeTok{fill=}\NormalTok{ rrhcare3)) }\OperatorTok{+}
\StringTok{  }\KeywordTok{geom_bar}\NormalTok{(}\DataTypeTok{stat =} \StringTok{"identity"}\NormalTok{) }\OperatorTok{+}
\StringTok{  }\KeywordTok{labs}\NormalTok{(}\DataTypeTok{x=} \StringTok{"White"}\NormalTok{,}
       \DataTypeTok{y=} \StringTok{"Percentage %"}\NormalTok{,}
       \DataTypeTok{fill =} \StringTok{"Experience"}\NormalTok{) }\OperatorTok{+}
\StringTok{  }\KeywordTok{scale_x_discrete}\NormalTok{(}\DataTypeTok{label =} \ControlFlowTok{function}\NormalTok{(x)}\KeywordTok{str_wrap}\NormalTok{(x, }\DataTypeTok{width =} \DecValTok{5}\NormalTok{))}\OperatorTok{+}
\StringTok{  }\KeywordTok{scale_y_continuous}\NormalTok{(}\DataTypeTok{labels =} \KeywordTok{percent_format}\NormalTok{(}\DataTypeTok{accuracy =} \DecValTok{1}\NormalTok{))}\OperatorTok{+}
\StringTok{  }\KeywordTok{geom_text}\NormalTok{(}
    \KeywordTok{aes}\NormalTok{(}
      \DataTypeTok{label=} \KeywordTok{paste0}\NormalTok{(}\KeywordTok{round}\NormalTok{((counts}\OperatorTok{/}\StringTok{ }\KeywordTok{sum}\NormalTok{(counts))}\OperatorTok{*}\DecValTok{100}\NormalTok{),}\StringTok{"%"}\NormalTok{),}
      \DataTypeTok{fontface=} \StringTok{"bold"}
\NormalTok{      ),}
    \DataTypeTok{vjust=} \DecValTok{2}
\NormalTok{    ) }\OperatorTok{+}
\StringTok{  }\KeywordTok{theme}\NormalTok{(}\DataTypeTok{axis.text.x =} \KeywordTok{element_text}\NormalTok{(}\DataTypeTok{hjust =} \FloatTok{0.5}\NormalTok{))}
\end{Highlighting}
\end{Shaded}

\includegraphics{Bias-hc_files/figure-latex/unnamed-chunk-9-1.pdf}

\hypertarget{for-the-black-or-african-american-race}{%
\subsubsection{For the black or african american
race}\label{for-the-black-or-african-american-race}}

\begin{Shaded}
\begin{Highlighting}[]
\NormalTok{race_black <-}\StringTok{ }\NormalTok{race_data1 }\OperatorTok
\StringTok{  }\KeywordTok{filter}\NormalTok{(rrclass2}\OperatorTok{==}\StringTok{"Black or African American"}\NormalTok{)}\OperatorTok
\StringTok{  }\KeywordTok{group_by}\NormalTok{(rrhcare3) }\OperatorTok
\StringTok{  }\KeywordTok{summarise}\NormalTok{(}\DataTypeTok{counts =} \KeywordTok{n}\NormalTok{())}
\end{Highlighting}
\end{Shaded}

\begin{Shaded}
\begin{Highlighting}[]
\NormalTok{race_black }\OperatorTok
\StringTok{  }\KeywordTok{ggplot}\NormalTok{(}\KeywordTok{aes}\NormalTok{(}\DataTypeTok{x=}\NormalTok{ rrhcare3,}
             \DataTypeTok{y=}\NormalTok{ counts}\OperatorTok{/}\KeywordTok{sum}\NormalTok{(counts),}
             \DataTypeTok{fill=}\NormalTok{ rrhcare3)) }\OperatorTok{+}
\StringTok{  }\KeywordTok{geom_bar}\NormalTok{(}\DataTypeTok{stat =} \StringTok{"identity"}\NormalTok{) }\OperatorTok{+}
\StringTok{  }\KeywordTok{labs}\NormalTok{(}\DataTypeTok{x=} \StringTok{"Black or African American"}\NormalTok{,}
       \DataTypeTok{y=} \StringTok{"Percentage %"}\NormalTok{,}
       \DataTypeTok{fill =} \StringTok{"Experience"}\NormalTok{) }\OperatorTok{+}
\StringTok{  }\KeywordTok{scale_x_discrete}\NormalTok{(}\DataTypeTok{label =} \ControlFlowTok{function}\NormalTok{(x)}\KeywordTok{str_wrap}\NormalTok{(x, }\DataTypeTok{width =} \DecValTok{5}\NormalTok{))}\OperatorTok{+}
\StringTok{  }\KeywordTok{scale_y_continuous}\NormalTok{(}\DataTypeTok{labels =} \KeywordTok{percent_format}\NormalTok{(}\DataTypeTok{accuracy =} \DecValTok{1}\NormalTok{))}\OperatorTok{+}
\StringTok{  }\KeywordTok{geom_text}\NormalTok{(}
    \KeywordTok{aes}\NormalTok{(}
      \DataTypeTok{label=} \KeywordTok{paste0}\NormalTok{(}\KeywordTok{round}\NormalTok{((counts}\OperatorTok{/}\StringTok{ }\KeywordTok{sum}\NormalTok{(counts))}\OperatorTok{*}\DecValTok{100}\NormalTok{),}\StringTok{"%"}\NormalTok{),}
      \DataTypeTok{fontface =} \StringTok{"bold"}
\NormalTok{      ),}
    \DataTypeTok{vjust =} \FloatTok{1.5}
\NormalTok{    ) }\OperatorTok{+}
\StringTok{  }\KeywordTok{theme}\NormalTok{(}\DataTypeTok{axis.text.x =} \KeywordTok{element_text}\NormalTok{(}\DataTypeTok{hjust =} \FloatTok{0.5}\NormalTok{))}
\end{Highlighting}
\end{Shaded}

\includegraphics{Bias-hc_files/figure-latex/unnamed-chunk-11-1.pdf}

As seen from the two graphs, the percentage of those having a better
healthcare experience than other races is higher in people of
\textbf{white} race\textbf{(15\%)} compared to those of \textbf{black or
african american} race\textbf{(10\%)}, but is this difference
statistically significant?

\begin{center}\rule{0.5\linewidth}{\linethickness}\end{center}

\begin{center}\rule{0.5\linewidth}{\linethickness}\end{center}

\begin{center}\rule{0.5\linewidth}{\linethickness}\end{center}

\hypertarget{data-analysis}{%
\section{Data Analysis}\label{data-analysis}}

I researched online on the question of using Chi-square vs logistic
regression in case of 2 categorical variables with 2 levels each. There
doesn't seem to be a straight answer, but I arrived at the conclusion
that both can be applied based on the question: Chi-square for
describing the strength of an association, and logistic regression for
modeling determinants and predicting the likelihood of an outcome.

I tested both methods here.

\hypertarget{performing-a-chi-square-test}{%
\subsubsection{Performing a Chi-square
test}\label{performing-a-chi-square-test}}

Using Chi-square test to test the strength of association between race
and healthcare experience

\begin{Shaded}
\begin{Highlighting}[]
\KeywordTok{chisq.test}\NormalTok{(tab1)}
\end{Highlighting}
\end{Shaded}

\begin{verbatim}
## 
##  Pearson's Chi-squared test with Yates' continuity correction
## 
## data:  tab1
## X-squared = 2.4746, df = 1, p-value = 0.1157
\end{verbatim}

\begin{Shaded}
\begin{Highlighting}[]
\KeywordTok{chisq.test}\NormalTok{(tab1, }\DataTypeTok{correct =}\NormalTok{ F)}
\end{Highlighting}
\end{Shaded}

\begin{verbatim}
## 
##  Pearson's Chi-squared test
## 
## data:  tab1
## X-squared = 2.8429, df = 1, p-value = 0.09178
\end{verbatim}

According to its results, the difference is \textbf{not statistically
significant}, so we fail to reject the null hypothesis of healthcare
experience and race being independent from each other.

\begin{center}\rule{0.5\linewidth}{\linethickness}\end{center}

\begin{center}\rule{0.5\linewidth}{\linethickness}\end{center}

\hypertarget{correlation-logistic-regression}{%
\subsubsection{Correlation \& Logistic
Regression:}\label{correlation-logistic-regression}}

Using logistic regression to predict the likelihood of better or same
healthcare experience based on race

\hypertarget{preparing-the-data}{%
\paragraph{Preparing the data}\label{preparing-the-data}}

\begin{Shaded}
\begin{Highlighting}[]
\NormalTok{logis_data <-}\StringTok{ }\NormalTok{race_data1 }\OperatorTok
\StringTok{  }\KeywordTok{mutate}\NormalTok{(}\DataTypeTok{race_dummy=}\NormalTok{ (}\KeywordTok{ifelse}\NormalTok{(rrclass2}\OperatorTok{==}\StringTok{"White"}\NormalTok{, }\DecValTok{0}\NormalTok{, }\DecValTok{1}\NormalTok{)),}
         \DataTypeTok{experience_dummy=}\NormalTok{ (}\KeywordTok{ifelse}\NormalTok{(rrhcare3}\OperatorTok{==}\StringTok{"The same as other races"}\NormalTok{, }\DecValTok{0}\NormalTok{, }\DecValTok{1}\NormalTok{)),}
         \DataTypeTok{rrclass2 =} \KeywordTok{as.factor}\NormalTok{(rrclass2),}
         \DataTypeTok{rrhcare3 =}  \KeywordTok{as.factor}\NormalTok{(rrhcare3))}

\NormalTok{logis_data <-}\StringTok{ }\KeywordTok{within}\NormalTok{(logis_data, rrclass2 <-}\StringTok{ }\KeywordTok{relevel}\NormalTok{(rrclass2, }\DataTypeTok{ref =} \StringTok{"White"}\NormalTok{))}
\end{Highlighting}
\end{Shaded}

\begin{center}\rule{0.5\linewidth}{\linethickness}\end{center}

\hypertarget{assessing-correlation-between-race-and-the-experience-of-seeking-healthcare}{%
\paragraph{Assessing correlation between race and the experience of
seeking
healthcare}\label{assessing-correlation-between-race-and-the-experience-of-seeking-healthcare}}

\begin{Shaded}
\begin{Highlighting}[]
\KeywordTok{cor}\NormalTok{(logis_data}\OperatorTok{$}\NormalTok{experience_dummy, logis_data}\OperatorTok{$}\NormalTok{race_dummy, }\DataTypeTok{method =} \StringTok{"spearman"}\NormalTok{)}
\end{Highlighting}
\end{Shaded}

\begin{verbatim}
## [1] -0.03270426
\end{verbatim}

\begin{Shaded}
\begin{Highlighting}[]
\KeywordTok{library}\NormalTok{(corrplot)}
\end{Highlighting}
\end{Shaded}

\begin{verbatim}
## corrplot 0.84 loaded
\end{verbatim}

\begin{Shaded}
\begin{Highlighting}[]
\NormalTok{correlation <-}\StringTok{ }\KeywordTok{cor}\NormalTok{(logis_data[,}\KeywordTok{c}\NormalTok{(}\DecValTok{3}\NormalTok{,}\DecValTok{4}\NormalTok{)])}
\KeywordTok{corrplot}\NormalTok{(correlation, }\DataTypeTok{method =} \StringTok{"circle"}\NormalTok{)}
\end{Highlighting}
\end{Shaded}

\includegraphics{Bias-hc_files/figure-latex/unnamed-chunk-16-1.pdf}

There is a \textbf{weak negative correlation} between race and the
experience of seeking healthcare.

\begin{center}\rule{0.5\linewidth}{\linethickness}\end{center}

\hypertarget{fitting-the-data-into-a-logistic-regression-model}{%
\paragraph{Fitting the data into a logistic regression
model}\label{fitting-the-data-into-a-logistic-regression-model}}

\begin{Shaded}
\begin{Highlighting}[]
\NormalTok{model1 <-}\StringTok{ }\KeywordTok{glm}\NormalTok{(}\DataTypeTok{formula =}\NormalTok{ rrhcare3}\OperatorTok{~}\StringTok{ }\NormalTok{rrclass2, }\DataTypeTok{family =}\NormalTok{ binomial, }\DataTypeTok{data =}\NormalTok{ logis_data)}
\KeywordTok{summary}\NormalTok{(model1)}
\end{Highlighting}
\end{Shaded}

\begin{verbatim}
## 
## Call:
## glm(formula = rrhcare3 ~ rrclass2, family = binomial, data = logis_data)
## 
## Deviance Residuals: 
##     Min       1Q   Median       3Q      Max  
## -2.1404   0.5672   0.5672   0.5672   0.5672  
## 
## Coefficients:
##                                   Estimate Std. Error z value Pr(>|z|)    
## (Intercept)                        1.74567    0.05634  30.984   <2e-16 ***
## rrclass2Black or African American  0.43840    0.26194   1.674   0.0942 .  
## ---
## Signif. codes:  0 '***' 0.001 '**' 0.01 '*' 0.05 '.' 0.1 ' ' 1
## 
## (Dispersion parameter for binomial family taken to be 1)
## 
##     Null deviance: 2206.1  on 2657  degrees of freedom
## Residual deviance: 2203.0  on 2656  degrees of freedom
## AIC: 2207
## 
## Number of Fisher Scoring iterations: 4
\end{verbatim}

\textbf{Notice} both Chi-square test and logistic regression model are
reporting very similar p- values, but only when continuity correction is
disabled in Chi-square.

\hypertarget{interpreting-the-model}{%
\paragraph{Interpreting the model}\label{interpreting-the-model}}

\begin{Shaded}
\begin{Highlighting}[]
\KeywordTok{exp}\NormalTok{(model1}\OperatorTok{$}\NormalTok{coefficients)}
\end{Highlighting}
\end{Shaded}

\begin{verbatim}
##                       (Intercept) rrclass2Black or African American 
##                          5.729730                          1.550222
\end{verbatim}

The \textbf{odds} of having a better healthcare experience than other
races if a person is \textbf{white} is \textbf{5.7}

The \textbf{odds} of having a better healthcare experience than other
races if a person is \textbf{black or african american} is \textbf{1.6}

\begin{Shaded}
\begin{Highlighting}[]
\NormalTok{p1 <-}\StringTok{ }\NormalTok{(}\KeywordTok{exp}\NormalTok{(}\FloatTok{1.74567}\NormalTok{)}\OperatorTok{/}\NormalTok{(}\DecValTok{1}\OperatorTok{+}\KeywordTok{exp}\NormalTok{(}\FloatTok{1.74567}\NormalTok{)))}
\NormalTok{p1}
\end{Highlighting}
\end{Shaded}

\begin{verbatim}
## [1] 0.8514058
\end{verbatim}

\begin{Shaded}
\begin{Highlighting}[]
\NormalTok{p2 <-}\StringTok{ }\NormalTok{(}\KeywordTok{exp}\NormalTok{(}\FloatTok{0.43840}\NormalTok{)}\OperatorTok{/}\NormalTok{(}\DecValTok{1}\OperatorTok{+}\KeywordTok{exp}\NormalTok{(}\FloatTok{0.43840}\NormalTok{)))}
\NormalTok{p2}
\end{Highlighting}
\end{Shaded}

\begin{verbatim}
## [1] 0.6078777
\end{verbatim}

The \textbf{probability} of having a better healthcare experience than
other races if a person is \textbf{white} is \textbf{85\%}

The \textbf{propability} of having a better healthcare experience than
other races if a person is \textbf{black or african american} is
\textbf{60\%}

\hypertarget{ploting-the-model}{%
\paragraph{Ploting the model}\label{ploting-the-model}}

I am not sure how to interpret the plots in this type of logistic
regression. I would appreciate any help.

\begin{Shaded}
\begin{Highlighting}[]
\KeywordTok{plot}\NormalTok{(model1)}
\end{Highlighting}
\end{Shaded}

\includegraphics{Bias-hc_files/figure-latex/unnamed-chunk-20-1.pdf}
\includegraphics{Bias-hc_files/figure-latex/unnamed-chunk-20-2.pdf}
\includegraphics{Bias-hc_files/figure-latex/unnamed-chunk-20-3.pdf}
\includegraphics{Bias-hc_files/figure-latex/unnamed-chunk-20-4.pdf}


\end{document}
